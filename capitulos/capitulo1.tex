\chapter{Introdução}\label{cap:introducao}

A introdução abre o trabalho propriamente dito. Tem a finalidade de apresentar os motivos que levaram o autor a realizar a pesquisa, o problema abordado, os objetivos e a justificativa. O objetivo principal da introdução é situar o leitor no contexto da pesquisa. O leitor deverá perceber claramente o que foi analisado, como e por que, as limitações encontradas, o alcance da investigação e suas bases teóricas gerais. Ela tem, acima de tudo, um caráter didático de apresentar o que foi investigado, levando-se em conta o leitor a que se destina e a finalidade do trabalho. 

Assim, na introdução contextualize o tema, delimite o assunto, apresente um rápido histórico do problema e das soluções porventura já apresentadas, com breve revisão crítica das investigações anteriores; faça referência às fontes de material, aos métodos seguidos, às teorias ou aos conceitos que embasam o desenvolvimento e a argumentação, às eventuais faltas de informação, ao instrumental utilizado. A introdução deverá conter, ainda:

\begin{enumerate}
   \item Justificativa;
   \item Definição do problema;
   \item Objetivo geral e objetivos específicos.
\end{enumerate}

\section{Objetivos}

\subsection{Objetivo geral}

\subsection{Objetivos específicos}


\section{Organização do texto}

O texto está organizado da seguinte forma: No \autoref{cap:revisao} é apresentado um pouco mais de como fazer um outro capítulo, apresentando ainda formas para inserir figuras. No \autoref{cap:proposta} é apresentado uma forma para adicionar uma tabela. Por fim, no \autoref{cap:conclusoes} são apresentadas as conclusões sobre este trabalho.