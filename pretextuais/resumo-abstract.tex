% ajusta o espaçamento dos parágrafos do resumo
\setlength{\absparsep}{18pt} 


\begin{resumo}
    O resumo deve ressaltar o objetivo, o método, os resultados e as conclusões do documento. A ordem e a extensão destes itens dependem do tipo de resumo (informativo ou indicativo) e do tratamento que cada item recebe no documento original. Utilize frases concisas e afirmativas. Explique o tema principal do trabalho na primeira frase. Use o verbo na voz ativa e na terceira pessoa do singular. 
    
    Palavras-chave: latex. abntex. editoração de texto.
\end{resumo}



%-----------------------------------------------%
\begin{resumo}[Abstract]
\begin{otherlanguage*}{english}
    This is the english abstract.
\vspace{\onelineskip}

\noindent 
Keywords: latex. abntex. text editoration.
\end{otherlanguage*}
\end{resumo}
%-----------------------------------------------%